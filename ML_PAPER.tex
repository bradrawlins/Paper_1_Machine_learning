%% 
%% Copyright 2019-2021 Elsevier Ltd
%% 
%% This file is part of the 'CAS Bundle'.
%% --------------------------------------
%% 
%% It may be distributed under the conditions of the LaTeX Project Public
%% License, either version 1.2 of this license or (at your option) any
%% later version.  The latest version of this license is in
%%    http://www.latex-project.org/lppl.txt
%% and version 1.2 or later is part of all distributions of LaTeX
%% version 1999/12/01 or later.
%% 
%% The list of all files belonging to the 'CAS Bundle' is
%% given in the file `manifest.txt'.
%% 
%% Template article for cas-dc documentclass for 
%% double column output.

\documentclass[a4paper,fleqn]{cas-dc}

% If the frontmatter runs over more than one page
% use the longmktitle option.

%\documentclass[a4paper,fleqn,longmktitle]{cas-dc}

%\usepackage[numbers]{natbib}
%\usepackage[authoryear]{natbib}
\usepackage[authoryear,longnamesfirst]{natbib}

%%%Author macros
\def\tsc#1{\csdef{#1}{\textsc{\lowercase{#1}}\xspace}}
\tsc{WGM}
\tsc{QE}
%%%

% Uncomment and use as if needed
%\newtheorem{theorem}{Theorem}
%\newtheorem{lemma}[theorem]{Lemma}
%\newdefinition{rmk}{Remark}
%\newproof{pf}{Proof}
%\newproof{pot}{Proof of Theorem \ref{thm}}

\begin{document}
\let\WriteBookmarks\relax
\def\floatpagepagefraction{1}
\def\textpagefraction{.001}

% Short title
%\shorttitle{<short title of the paper for running head>}    

% Short author
\shortauthors{B.T. Rawlins et el.}  

% Main title of the paper
\title [mode = title]{This is a Title}  

% Title footnote mark
% eg: \tnotemark[1]
%\tnotemark[<tnote number>] 

% Title footnote 1.
% eg: \tnotetext[1]{Title footnote text}
%\tnotetext[<tnote number>]{<tnote text>} 

% First author
%
% Options: Use if required
%\author[1,3]{Author Name}[type=editor,
%       style=chinese,
%       auid=000,
%       bioid=1,
%       prefix=Sir,
%       orcid=0000-0000-0000-0000,
%       facebook=<facebook id>,
%       twitter=<twitter id>,
%       linkedin=<linkedin id>,
%       gplus=<gplus id>]

\author[1]{B.T. Rawlins}

% Corresponding author indication
\cormark[1]
\cortext[1]{Corresponding author}
% Footnote of the first author
%\fnmark[<footnote mark no>]

% Email id of the first author
\ead{rwlbra001@myuct.ac.za}

% URL of the first author
%\ead[url]{<URL>}

% Credit authorship
% eg: \credit{Conceptualization of this study, Methodology, Software}
\credit{Conceptualization of this study, Methodology, Software}

% Address/affiliation
\affiliation[1]{organization={Department of Mechanical Engineering, Applied Thermal-Fluid Process Modelling Research Unit, University of Cape Town},
            addressline={Library Road, Rondebosch}, 
            city={Cape Town},
%          citysep={}, % Uncomment if no comma needed between city and postcode
            postcode={7701}, 
            %state={},
            country={South Africa}}

\author[2]{Ryno Laubscher}[]

% Footnote of the second author
%\fnmark[2]

% Email id of the second author
\ead{rlaubscher@sun.ac.za}
% URL of the second author
%\ead[url]{}
% Credit authorship
\credit{Review}
% Address/affiliation
\affiliation[2]{organization={Department of Mechanical Engineering, Stellenbosch University},
            addressline={Banghoek Road, Stellenbosch}, 
            %city={Stellenbosch},
%          citysep={}, % Uncomment if no comma needed between city and postcode
            postcode={7600}, 
            %state={},
            country={South Africa}}
% Footnote text
%\fntext[1]{}
% For a title note without a number/mark
%\nonumnote{}
\author[1]{Pieter Rousseau}
% Email id of the second author
\ead{pieter.rousseau@uct.ac.za}
% Credit authorship
\credit{Review}

% Corresponding author text


% Here goes the abstract
\begin{abstract}
ss fhvosjv \\
sfjvnsfklnvjksfn\\
skfnvkpsfn \\
sjkvnsf skfnv\\
ksfvn;sfv,  
\end{abstract}

% Use if graphical abstract is present
%\begin{graphicalabstract}
%\includegraphics{}
%\end{graphicalabstract}

% Research highlights
\begin{highlights}
\item Computational Fluid Dynamics capture trend of boiler heat uptake at reduced loads.
\item Computational Fluid Dynamics capture trend of boiler heat uptake at reduced loads.
\item Computational Fluid Dynamics capture trend of boiler heat uptake at reduced loads.
\end{highlights}

% Keywords
% Each keyword is seperated by \sep
\begin{keywords}
Mixture density model \sep Steady State \sep \sep
\end{keywords}

\maketitle

% Main text
\section{Introduction}\label{intro}

% Numbered list
% Use the style of numbering in square brackets.
% If nothing is used, default style will be taken.
%\begin{enumerate}[a)]
%\item 
%\item 
%\item 
%\end{enumerate}  

% Unnumbered list
%\begin{itemize}
%\item 
%\item 
%\item 
%\end{itemize}  

% Description list
%\begin{description}
%\item[]
%\item[] 
%\item[] 
%\end{description}  

% Figure
%\begin{figure}%[<options>]
	%\centering
	%	\includegraphics[<options>]{}
	 % \caption{}\label{fig1}
%\end{figure}


%\begin{table}[<options>]
%\caption{}\label{tbl1}
%\begin{tabular*}{\tblwidth}{@{}LL@{}}
%\toprule
 % &  \\ % Table header row
%\midrule
 %& \\
 %& \\
 %& \\
% & \\
%\bottomrule
%\end{tabular*}
%\end{table}

% Uncomment and use as the case may be
%\begin{theorem} 
%\end{theorem}

% Uncomment and use as the case may be
%\begin{lemma} 
%\end{lemma}

%% The Appendices part is started with the command \appendix;
%% appendix sections are then done as normal sections
%% \appendix

\section{Mathematical Model}

\subsection{Artificial neural networks}

\subsection{Mixture density networks}

\section{Case study boiler configuration}

\section{Model development}

\subsection{Hyper parameter tuning}

\section{Results and discussion}

\section{Conclusion}
% To print the credit authorship contribution details
%\printcredits

%% Loading bibliography style file
%\bibliographystyle{model1-num-names}
\bibliographystyle{cas-model2-names}

% Loading bibliography database
\bibliography{}

% Biography
\bio{}
% Here goes the biography details.
\endbio

%$\bio{pic1}
% Here goes the biography details.
%\endbio

\end{document}

