%% 
%% Copyright 2019-2021 Elsevier Ltd
%% 
%% This file is part of the 'CAS Bundle'.
%% --------------------------------------
%% 
%% It may be distributed under the conditions of the LaTeX Project Public
%% License, either version 1.2 of this license or (at your option) any
%% later version.  The latest version of this license is in
%%    http://www.latex-project.org/lppl.txt
%% and version 1.2 or later is part of all distributions of LaTeX
%% version 1999/12/01 or later.
%% 
%% The list of all files belonging to the 'CAS Bundle' is
%% given in the file `manifest.txt'.
%% 
%% Template article for cas-sc documentclass for 
%% single column output.

\documentclass[a4paper,fleqn]{cas-sc}
% Use to make nonmenclature
\usepackage{framed} % Framing content

\usepackage{multicol} % Multiple columns environment
\usepackage{amsmath}
\usepackage{nomencl} % Nomenclature package

\makenomenclature

\setlength{\nomitemsep}{-\parskip} % Baseline skip between items

\renewcommand*\nompreamble{\begin{multicols}{2}}

\renewcommand*\nompostamble{\end{multicols}}
% If the frontmatter runs over more than one page
% use the longmktitle option.

%\documentclass[a4paper,fleqn,longmktitle]{cas-sc}

\usepackage[numbers]{natbib}
%\usepackage[authoryear]{natbib}
%\usepackage[authoryear,longnamesfirst]{natbib}

%%%Author macros
\def\tsc#1{\csdef{#1}{\textsc{\lowercase{#1}}\xspace}}
\tsc{WGM}
\tsc{QE}
%%%

% Uncomment and use as if needed
%\newtheorem{theorem}{Theorem}
%\newtheorem{lemma}[theorem]{Lemma}
%\newdefinition{rmk}{Remark}
%\newproof{pf}{Proof}
%\newproof{pot}{Proof of Theorem \ref{thm}}

\begin{document}
\let\WriteBookmarks\relax
\def\floatpagepagefraction{1}
\def\textpagefraction{.001}

% Short title
\shorttitle{}    

% Short author
\shortauthors{B.T. Rawlins et el.}  

% Main title of the paper
\title [mode = title]{Validation of an integrated data-driven surrogate model and a thermo-hydraulic network based model to determine boiler operational loads using a fully connected mixture density network}  

% Title footnote mark
% eg: \tnotemark[1]
%\tnotemark[<tnote number>] 

% Title footnote 1.
% eg: \tnotetext[1]{Title footnote text}
%\tnotetext[<tnote number>]{<tnote text>} 

% First author
%
% Options: Use if required
%\author[1,3]{Author Name}[type=editor,
%       style=chinese,
%       auid=000,
%       bioid=1,
%       prefix=Sir,
%       orcid=0000-0000-0000-0000,
%       facebook=<facebook id>,
%       twitter=<twitter id>,
%       linkedin=<linkedin id>,
%       gplus=<gplus id>]

\author[1]{B.T. Rawlins}

% Corresponding author indication
\cormark[1]
\cortext[1]{Corresponding author}
% Footnote of the first author
%\fnmark[<footnote mark no>]

% Email id of the first author
\ead{rwlbra001@myuct.ac.za}

% URL of the first author
%\ead[url]{<URL>}

% Credit authorship
% eg: \credit{Conceptualization of this study, Methodology, Software}
\credit{Methodology, Software, Validation, Formal analysis, Investigation,Writing original draft, Visualization.}

% Address/affiliation
\affiliation[1]{organization={Department of Mechanical Engineering, Applied Thermal-Fluid Process Modelling Research Unit, University of Cape Town},
            addressline={Library Road, Rondebosch}, 
            city={Cape Town},
%          citysep={}, % Uncomment if no comma needed between city and postcode
            postcode={7701}, 
            %state={},
            country={South Africa}}

\author[2]{Ryno Laubscher}[]

% Footnote of the second author
%\fnmark[2]

% Email id of the second author
\ead{rlaubscher@sun.ac.za}
% URL of the second author
%\ead[url]{}
% Credit authorship
\credit{Writing review \& editing, Methodology, Resources, Conceptualization.}
% Address/affiliation
\affiliation[2]{organization={Department of Mechanical Engineering, Stellenbosch University},
            addressline={Banghoek Road, Stellenbosch}, 
            %city={Stellenbosch},
%          citysep={}, % Uncomment if no comma needed between city and postcode
            postcode={7600}, 
            %state={},
            country={South Africa}}
% Footnote text
%\fntext[1]{}
% For a title note without a number/mark
%\nonumnote{}
\author[1]{Pieter Rousseau}
% Email id of the second author
\ead{pieter.rousseau@uct.ac.za}
% Credit authorship
\credit{Writing review \& editing, Resources, Conceptualization}

% Corresponding author text


% Here goes the abstract
\begin{abstract}
A data-driven surrogate model is proposed for a 620$MW_e$ sub-critical power boiler. The surrogate model was developed using computational fluid dynamic (CFD) simulation data. The simulation data covered a varied range of inputs.
\end{abstract}

% Use if graphical abstract is present
%\begin{graphicalabstract}
%\includegraphics{}
%\end{graphicalabstract}

% Research highlights
\begin{highlights}
\item Development of mixture density network using simulation data.
\item Model based on validated CFD model of a 620 $MW_e$ sub-critical boiler.
\item Surrogate model prediction errors are below 10\%.
\end{highlights}

% Keywords
% Each keyword is seperated by \sep
\begin{keywords}
Mixture density network \sep Surrogate modelling \sep Boiler operation \sep 
\end{keywords}

\maketitle

% Main text
\section{Introduction}\label{intro}

The use of neural networks for the modelling of energy systems has been awesome. Optimization of a plant is extremely fun


\newpage
\begin{table*}[!t]   

\begin{framed}

\nomenclature{$abbreviation$}{explanation for the abbreviation}
\nomenclature{$CFD$}{Computational Fluid Dynamics}
\nomenclature{$u,\,\,\,[m/s]$}{Directional velocity}
\nomenclature{$p,\,\,\,[Pa]$}{Pressure}
\nomenclature{$E,\,\,\,[J/kg]$}{Total energy}
\nomenclature{$Y_k,\,\,\,[kg/kg]$}{Mass fraction of species $k$}
\nomenclature{$T_g,\,\,\,[K]$}{Gas temperature}
\nomenclature{$\lambda,\,\,\,[W/mK]$}{Thermal conductivity}
\nomenclature{$\mu,\,\,\,[Pa.s]$}{Viscosity}
\nomenclature{$S,\,\,\,[kg/m^3]$}{Mass source term}
\nomenclature{$S_m,\,\,\,[N/m^3]$}{Momentum source term}
\nomenclature{$S_h,\,\,\,[W/m^3]$}{Energy source term}
\nomenclature{$S_k,\,\,\,[kg/m^3]$}{Species source term}
\nomenclature{$J_k,\,\,\,[kg/m^3]$}{Diffusion flux of species $k$}
\printnomenclature

\end{framed}

\end{table*}
% Numbered list
% Use the style of numbering in square brackets.
% If nothing is used, default style will be taken.
%\begin{enumerate}[a)]
%\item 
%\item 
%\item 
%\end{enumerate}  

% Unnumbered list
%\begin{itemize}
%\item 
%\item 
%\item 
%\end{itemize}  

% Description list
%\begin{description}
%\item[]
%\item[] 
%\item[] 
%\end{description}  

% Figure
%\begin{figure}%[<options>]
	%\centering
	%	\includegraphics[<options>]{}
	 % \caption{}\label{fig1}
%\end{figure}


%\begin{table}[<options>]
%\caption{}\label{tbl1}
%\begin{tabular*}{\tblwidth}{@{}LL@{}}
%\toprule
 % &  \\ % Table header row
%\midrule
 %& \\
 %& \\
 %& \\
% & \\
%\bottomrule
%\end{tabular*}
%\end{table}

% Uncomment and use as the case may be
%\begin{theorem} 
%\end{theorem}

% Uncomment and use as the case may be
%\begin{lemma} 
%\end{lemma}

%% The Appendices part is started with the command \appendix;
%% appendix sections are then done as normal sections
%% \appendix

\section{Applicable machine learning theory}

\section{Data generation}

A steady-state multiphase non-thermal equilibrium CFD model was used to generate the target data, and was subsequently used for training/development of an appropriate machine learning model.

\subsection{CFD model setup}

The current study makes use of the commercial CFD software package ANSYS\textsuperscript{\textregistered} Fluent 2019 R3 to resolve the fluid flow, heat transfer and combustion processes for 620$MW_e$ utility scale boiler burning a pulverised fuel. The computational domain is modelled on a symmetry plane half way through the depth of the boiler. This was done to reduce the cell count. Figure () highlights the computational domain and illustrates the important boundary conditions.\\ 

(INSERT FIGURE)

The general conservation equations, which include, continuity, momentum, energy and species, were solved using a Eulerian approach. The subsequent equations can be seen in Equation (\ref{eqn_cfd}).

\begin{flalign} \label{eqn_cfd}
&\frac{\partial}{\partial x_{i}}(\rho \bar{u}_{i})=S \nonumber &&\\
&\frac{\partial}{\partial x_{i}}(\rho_{eff} u_{i}u_{j})+\frac{\partial \overline{p}}{\partial x_{j}}=\frac{\partial}{\partial x_{i}}\left[\mu\left\{\frac{\partial u_{j}}{\partial x_{i}}+\frac{\partial u_{i}}{\partial x_{j}}-\frac{2}{3}\delta_{ij}\frac{\partial u_{i}}{\partial x_{i}}\right\}\right]+\frac{\partial}{\partial x_{i}}(-\rho\overline{u_{i}^{'}u_{j}^{'}})+S_m \nonumber &&\\
&\frac{\partial }{\partial x_{i}} (u_{i}[\rho E+p])=\frac{\partial }{\partial x_{j}}\left[\lambda\frac{\partial T_{g}}{\partial x_{j}}\right] +S_{h} &&\\
&\frac{\partial}{\partial x_{i}}(\rho u_{j}Y_{k})=-\frac{\partial}{\partial x_{j}}(\vec{J_{k}})+ \sum_r R_{j,r} + S_{k} \nonumber && 
\end{flalign}

The resolution of the Reynolds stress term found in the momentum equation, $-\rho\overline{u_{i}^{'}u_{j}^{'}}$, was approximated using the Boussineq equation \citep{Versteeg2007}. In the present study the realizable k-$\varepsilon$ turbulence model was utilized to address the turbulence closure problem, this model was selected for its applicability in modelling the effects of coal-fired swirl burners \citep{Modlinski2010}.\\

The P1 radiation model was used to resolve the radiative field in the domain. Particle transport was modelled using a multiphase approach with further details on the modelling is provided in the validation study of Rawlins et al \citep{Rawlins2021}. The combustion follows a four step sequential process, beginning with the heating and evaporation of moisture in the fuel, followed by devolatilization of the volatiles, the phenomena of char burnout would follow, finally the gas phase reactions can commence. The char oxidation reaction was set so that the product species is $CO$. For the gas-phase reactions the turbulence-chemistry interaction was approximated using the eddy dissipation model. A summary of the combustion equations and constants are provided in Table \ref{tbl_combust} for the interested reader.\\

\begin{table}[h!]
\caption{Summary of combustion models and constants used in the CFD model}\label{tbl_combust}
\begin{tabular*}{\tblwidth}{p{0.35\textwidth}p{0.35\textwidth}p{0.25\textwidth}}
\toprule
Model & Equation/s & Constant/s\\
\midrule
\multicolumn{3}{l}{\textit{Devolatilization}} \\ % Table header row
Single rate kinetic &$\frac{dm_{vol}}{dt} = R_{vol}(m_{0,vol}-m_{vol})$, $R_{vol} = A_{vol}exp\left(\frac{E_{a,vol}}{RT_p}\right)$  & $A_{vol} = 2\times10^5 [s^{-1}]$, $ E_{a,vol} = 6.7\times10^7 [J/kmol]$ - \cite{Sheng2004} \\
\multicolumn{3}{l}{\textit{Char oxidation}} \\
Diffusion/kinetic - \citep{Baum1971} & $\frac{dm_{char}}{dt} = -A_p p_{O_{2}} \frac{R_{diff}R_c}{R_{diff} + R_c}$, $R_{c} = A_{c}exp\left(\frac{E_{a,c}}{RT_p}\right)$, $R_{diff} = \frac{5\times10^{-12}}{d_p} \left(\frac{T_g+T_p}{2}\right)^{0.75}$ & $A_{c} = 0.0053 [kg/m^2sPa]$, $E_{a,c} = 8.37\times10^7 [J/kmol]$ - \cite{Sheng2004}\\
\multicolumn{3}{l}{\textit{Gaseous reactions of volatiles and $CO$}} \\
Eddy dissipation model - \cite{Ansys} & $R_{k,r,P} =\vartheta_{k,r}M_{w,k}AB\rho\frac{\varepsilon}{k}min\left(\frac{\sum_{p} Y_p}{\sum_{j}\vartheta_{j,r}M_{w,j}}\right)$, $R_{k,r,R} =\vartheta_{k,r}M_{w,k}A\rho\frac{\varepsilon}{k}min\left(\frac{Y_R}{\varepsilon_{R,r}M_{w,R}}\right)$ & $A=4.0$, $B=0.5$\\
\bottomrule
\end{tabular*}
\end{table}

The simulations were solved using the SIMPLE pressure–velocity coupling scheme. The pressure term 
was discretized using the PRESTO! scheme. Momentum, species and energy equations were discretized 
using the second-order upwind scheme and the turbulent kinetic energy and dissipation rate using
the first-order upwind scheme. The convergence criteria for the simulation model was set to $1\times10^{-3}$ for the continuity equation, $1\times10^{-4}$ for the velocity equations, $1\times10^{-6}$ for the remaining transport equations, and $1\times10^{-4}$ for monitored key parameters.

\subsection{Simulated dataset}
design of experiments (DOE) was conducted to generate a set of 180 simulation cases.

Use inputs based on 

High low values for DOE what mapping etc

\begin{table}[h!]
\caption{Design of experiments input ranges for  simulations}\label{tbl_doe}
\begin{tabular*}{\tblwidth}{p{0.5\textwidth}p{0.15\textwidth}p{0.15\textwidth}p{0.15\textwidth}}
\toprule
 Input variable& Min& Max& Units \\ % Table header row
\midrule
 Fuel flow rate for mills 1 to 6 & & & $kg/s$ \\
 Fuel proximate analysis moisture mass fraction, $Y_{H_2O}$ & 0.025 & 0.085 & $kg/kg$ \\
 Fuel proximate analysis ash mass fraction, $Y_{ash}$  & 0.259 & 0.559 & $kg/kg$ \\
 Platen SH fouling thermal resistance, $R_{platen}$  & 0.004 & 0.007 & $m^2K/W$ \\
 Final SH fouling thermal resistance, $R_{final}$  &0.01 & 0.017 & $m^2K/W$ \\
\bottomrule
\end{tabular*}
\end{table}

\section{Model development}

The present work makes use of two types of machine learning models, namely a standard artificial neural network (ANN) and a mixture density designated model connected to a standard ANN (MDN-ANN). The following section will discuss the hyper parameter tuning and final selected model configuration. The programming language Python 3.7.8 and the Tensorflow machine learning libraries were utilized in the present study. 

\subsection{Data pre-processing}


\subsection{Hyper parameter tuning \& final model selection}

table of NN and MDN data comparison for tuning

\clearpage
\begin{table}[h!]
\caption{Hyperparameter search space for fully connected NN and MDN models}\label{tbl_tuning}
\begin{tabular*}{\tblwidth}{p{0.5\textwidth}p{0.24\textwidth}p{0.24\textwidth}}
\toprule
 Parameter& NN search space & MDN search space \\ % Table header row
\midrule
 Number of distributions & - & 2,3,4  \\
 Number of neurons per layer & 10, 40, 80, 100  & 10, 40, 80, 100\\
 Learning rates & 1e-3, 1e-4, 1e-5 &  1e-3, 1e-4, 1e-5   \\
 Mini batch sizes  & 16, 32, 64 & 16, 32, 64  \\
\bottomrule
\end{tabular*}
\end{table}





\section{Results and discussion}

\section{Conclusion}

The present work has shown it is possible
% To print the credit authorship contribution details
\printcredits

%% Loading bibliography style file
%\bibliographystyle{model1-num-names}
\bibliographystyle{cas-model2-names}

% Loading bibliography database
\bibliography{ML_paper}

% Biography
\bio{}
% Here goes the biography details.
\endbio

%$\bio{pic1}
% Here goes the biography details.
%\endbio

\end{document}


