%% 
%% Copyright 2019-2021 Elsevier Ltd
%% 
%% This file is part of the 'CAS Bundle'.
%% --------------------------------------
%% 
%% It may be distributed under the conditions of the LaTeX Project Public
%% License, either version 1.2 of this license or (at your option) any
%% later version.  The latest version of this license is in
%%    http://www.latex-project.org/lppl.txt
%% and version 1.2 or later is part of all distributions of LaTeX
%% version 1999/12/01 or later.
%% 
%% The list of all files belonging to the 'CAS Bundle' is
%% given in the file `manifest.txt'.
%% 
%% Template article for cas-sc documentclass for 
%% single column output.

\documentclass[a4paper,fleqn]{cas-sc}
% Use to make nonmenclature
\usepackage{framed} % Framing content

\usepackage{multicol} % Multiple columns environment

\usepackage{nomencl} % Nomenclature package

\makenomenclature

\setlength{\nomitemsep}{-\parskip} % Baseline skip between items

\renewcommand*\nompreamble{\begin{multicols}{2}}

\renewcommand*\nompostamble{\end{multicols}}
% If the frontmatter runs over more than one page
% use the longmktitle option.

%\documentclass[a4paper,fleqn,longmktitle]{cas-sc}

\usepackage[numbers]{natbib}
%\usepackage[authoryear]{natbib}
%\usepackage[authoryear,longnamesfirst]{natbib}

%%%Author macros
\def\tsc#1{\csdef{#1}{\textsc{\lowercase{#1}}\xspace}}
\tsc{WGM}
\tsc{QE}
%%%

% Uncomment and use as if needed
%\newtheorem{theorem}{Theorem}
%\newtheorem{lemma}[theorem]{Lemma}
%\newdefinition{rmk}{Remark}
%\newproof{pf}{Proof}
%\newproof{pot}{Proof of Theorem \ref{thm}}

\begin{document}
\let\WriteBookmarks\relax
\def\floatpagepagefraction{1}
\def\textpagefraction{.001}

% Short title
\shorttitle{}    

% Short author
\shortauthors{B.T. Rawlins et el.}  

% Main title of the paper
\title [mode = title]{Validation of an integrated data-driven surrogate model and a thermo-hydraulic network based model to determine boiler operational loads using a fully connected mixture density network}  

% Title footnote mark
% eg: \tnotemark[1]
%\tnotemark[<tnote number>] 

% Title footnote 1.
% eg: \tnotetext[1]{Title footnote text}
%\tnotetext[<tnote number>]{<tnote text>} 

% First author
%
% Options: Use if required
%\author[1,3]{Author Name}[type=editor,
%       style=chinese,
%       auid=000,
%       bioid=1,
%       prefix=Sir,
%       orcid=0000-0000-0000-0000,
%       facebook=<facebook id>,
%       twitter=<twitter id>,
%       linkedin=<linkedin id>,
%       gplus=<gplus id>]

\author[1]{B.T. Rawlins}

% Corresponding author indication
\cormark[1]
\cortext[1]{Corresponding author}
% Footnote of the first author
%\fnmark[<footnote mark no>]

% Email id of the first author
\ead{rwlbra001@myuct.ac.za}

% URL of the first author
%\ead[url]{<URL>}

% Credit authorship
% eg: \credit{Conceptualization of this study, Methodology, Software}
\credit{Methodology, Software, Validation, Formal analysis, Investigation,Writing original draft, Visualization.}

% Address/affiliation
\affiliation[1]{organization={Department of Mechanical Engineering, Applied Thermal-Fluid Process Modelling Research Unit, University of Cape Town},
            addressline={Library Road, Rondebosch}, 
            city={Cape Town},
%          citysep={}, % Uncomment if no comma needed between city and postcode
            postcode={7701}, 
            %state={},
            country={South Africa}}

\author[2]{Ryno Laubscher}[]

% Footnote of the second author
%\fnmark[2]

% Email id of the second author
\ead{rlaubscher@sun.ac.za}
% URL of the second author
%\ead[url]{}
% Credit authorship
\credit{Writing review \& editing, Methodology, Resources, Conceptualization.}
% Address/affiliation
\affiliation[2]{organization={Department of Mechanical Engineering, Stellenbosch University},
            addressline={Banghoek Road, Stellenbosch}, 
            %city={Stellenbosch},
%          citysep={}, % Uncomment if no comma needed between city and postcode
            postcode={7600}, 
            %state={},
            country={South Africa}}
% Footnote text
%\fntext[1]{}
% For a title note without a number/mark
%\nonumnote{}
\author[1]{Pieter Rousseau}
% Email id of the second author
\ead{pieter.rousseau@uct.ac.za}
% Credit authorship
\credit{Writing review \& editing, Resources, Conceptualization}

% Corresponding author text


% Here goes the abstract
\begin{abstract}
A data-driven surrogate model is proposed for a 620$MW_e$ sub-critical power boiler. The surrogate model was developed using computational fluid dynamic (CFD) simulation data. The simulation data covered a varied range of inputs.
\end{abstract}

% Use if graphical abstract is present
%\begin{graphicalabstract}
%\includegraphics{}
%\end{graphicalabstract}

% Research highlights
\begin{highlights}
\item Development of mixture density network using simulation data.
\item Model based on validated CFD model of a 620 $MW_e$ sub-critical boiler.
\item Surrogate model prediction errors are below 10\%.
\end{highlights}

% Keywords
% Each keyword is seperated by \sep
\begin{keywords}
Mixture density network \sep Surrogate modelling \sep Boiler operation \sep 
\end{keywords}

\maketitle

% Main text
\section{Introduction}\label{intro}

The use of neural networks for the modelling of energy systems has been awesome. Optimization of a plant is extremely fun


\newpage
\begin{table*}[!t]   

\begin{framed}

\nomenclature{$abbreviation$}{explanation for the abbreviation}
\nomenclature{$CFD$}{Computational Fluid Dynamics}
\printnomenclature

\end{framed}

\end{table*}
% Numbered list
% Use the style of numbering in square brackets.
% If nothing is used, default style will be taken.
%\begin{enumerate}[a)]
%\item 
%\item 
%\item 
%\end{enumerate}  

% Unnumbered list
%\begin{itemize}
%\item 
%\item 
%\item 
%\end{itemize}  

% Description list
%\begin{description}
%\item[]
%\item[] 
%\item[] 
%\end{description}  

% Figure
%\begin{figure}%[<options>]
	%\centering
	%	\includegraphics[<options>]{}
	 % \caption{}\label{fig1}
%\end{figure}


%\begin{table}[<options>]
%\caption{}\label{tbl1}
%\begin{tabular*}{\tblwidth}{@{}LL@{}}
%\toprule
 % &  \\ % Table header row
%\midrule
 %& \\
 %& \\
 %& \\
% & \\
%\bottomrule
%\end{tabular*}
%\end{table}

% Uncomment and use as the case may be
%\begin{theorem} 
%\end{theorem}

% Uncomment and use as the case may be
%\begin{lemma} 
%\end{lemma}

%% The Appendices part is started with the command \appendix;
%% appendix sections are then done as normal sections
%% \appendix

\section{Data generation}
This section aims to highlight the important and relevant theory behind ANN

\subsection{CFD model setup}
The CFD mdoel is from the passt paper data is good

Validated for variable load range as shown by Rawlins et al()

\subsection{Simulated dataset}

\section{Model development}

\subsection{Network based model integration}


\subsection{Hyper parameter tuning}

table of NN and MDN data comparison for tuning

\clearpage
\begin{table}[h!]
\caption{Hyperparameter search space for fully connected NN and MDN models}\label{tbl_tuning}
\begin{tabular*}{\tblwidth}{p{0.5\textwidth}p{0.24\textwidth}p{0.24\textwidth}}
\toprule
 Parameter& NN search space & MDN search space \\ % Table header row
\midrule
 Number of distributions & - & 2,3,4  \\
 Number of neurons per layer & 10, 40, 80, 100  & 10, 40, 80, 100\\
 Learning rates & 1e-3, 1e-4, 1e-5 &  1e-3, 1e-4, 1e-5   \\
 Mini batch sizes  & 16, 32, 64 & 16, 32, 64  \\
\bottomrule
\end{tabular*}
\end{table}



\begin{table}[h!]
\caption{Design of experiments input ranges for  simulations}\label{tbl_doe}
\begin{tabular*}{\tblwidth}{p{0.5\textwidth}p{0.15\textwidth}p{0.15\textwidth}p{0.15\textwidth}}
\toprule
 Input variable& Min& Max& Units \\ % Table header row
\midrule
 Fuel flow rate for mills 1 to 6 & & & $kg/s$ \\
 Fuel proximate analysis moisture mass fraction, $Y_{H_2O}$ & 0.025 & 0.085 & $kg/kg$ \\
 Fuel proximate analysis ash mass fraction, $Y_{ash}$  & 0.259 & 0.559 & $kg/kg$ \\
 Platen SH fouling thermal resistance, $R_{platen}$  & 0.004 & 0.007 & $m^2K/W$ \\
 Final SH fouling thermal resistance, $R_{final}$  &0.01 & 0.017 & $m^2K/W$ \\
\bottomrule
\end{tabular*}
\end{table}

\section{Results and discussion}

\section{Conclusion}

The present work has shown it is possible
% To print the credit authorship contribution details
\printcredits

%% Loading bibliography style file
%\bibliographystyle{model1-num-names}
\bibliographystyle{cas-model2-names}

% Loading bibliography database
\bibliography{}

% Biography
\bio{}
% Here goes the biography details.
\endbio

%$\bio{pic1}
% Here goes the biography details.
%\endbio

\end{document}


